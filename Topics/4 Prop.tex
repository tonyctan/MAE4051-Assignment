\begin{center}
    \textbf{Propensity Scores}
\end{center}

\begin{hangparas}{0.5in}{1}
    \fullcite{sullivan:2013}
\end{hangparas}

\section{Summary}
%//mark Summary of the article

\textcite{sullivan:2013} attempted to investigate the marginal benefit created by a special education program mandated by the US \textit{Individuals with Disabilities Education Act} (IDEA). The 1986 amendments to the IDEA legislation imposed legal requirement on the states to create a sequence of intervention programs targeting three stages of development for children with needs: 1) from birth to 2-year-old, 2) 3-year to kindergarten entry, and 3) kindergarten to 21-year-and-11-month-old. This paper focused on the middle segment of the intervention sequence named preschool special education services and followed children longitudinally through the delivery period. The study began by measuring children's probability of being admitted into the special education program using a logit regression, and at the end of the intervention period, compared the average performance scores in maths and reading of the children who actually received the preschool special education treatment against the average scores of those who have not. Using the propensity score weighting technique, the authors interrogated the counterfactual question of ``what if'' the treatment were never applied and reached an conclusion that children in treatment group would have been better off academically had they not received any intervention at all. This disappointing result lent itself to an existing literature of similar negative findings evaluating special education effectiveness, cautioning the rosy objectives of preschool special education services originally marketed by policy makers.

\section{Causal Question}
%//mark Identification of the causal question investigated in the study

Would the children who received special education services have been better off academically, on average, had they not received such services?

\section{Validity}
%//mark Discussion of validity aspects of the study: statistical conclusion validity, internal validity, construct validity, and external validity

\subsection{Internal and External Validity}

Historically, internal validity referred to inferences about whether ``the experimental treatments make a difference in this specific experimental instance'' while external validity asked ``to what populations, settings, treatment variables, and measurement variables can this effect be generalized'' \parencite[][p. 5]{campbell:1963}. \textcite{cook:1979} advanced the idea of internal validity to the question whether the covariation observed between the independent and dependent variables were resulted from a \emph{causal} relationship, whereas external validity further asks whether such cause-effect relationship holds over certain variation in persons, settings, treatment variables, and measurement variables.

In order to support an inference that the observed covariation between $A$ and $B$ reflects a causal relationship, \textcite{shadish:2002} prescribed a trifecta that 1) $A$ preceded $B$ in time, 2) $A$ covaries with $B$, and 3) no other explanations for the relationship are plausible. It is too often the third strand that undermines the internal validity of inference making---the relationship between $A$ and $B$ is not causal because it could have occurred even in the absence of the treatment and that it could have led to the same outcomes that were observed for the treatment. Amongst the list of potential threats to internal validity \parencite[][pp. 54--61]{shadish:2002}, maturation presents the strongest challenge to \textcite{sullivan:2013}. Children who have been identified as ``in need'' so early in life (``early onsetters'') can be reasonably believed to be in possession of different developmental profiles from children who showed needs later in life (``late onsetters''). As participants in the treatment group mature, gaps in academic performance may well emerge out of such delayed developmental trajectories with or without education services. It is therefore not preschool interventions that ``caused'' lower academic scores but the two-tier growth profiles that did. Failing to rule out such alternative, and rather plausible, explanation weakens the internal validity of inference made by the authors.

Weak external validity has also been acknowledged by the authors in Section 4.2 of the paper. Inferences can only been drawn, first of all, over children with mild to moderate impairments resultant from the sample deletion procedure; while it were the children with the most severe impairment that policy makers wished to monitor and retain (``interaction of the causal relationship with units'' by \textcite{shadish:2002}). Secondly, \textcite{sullivan:2013} evaluated only the academic performance of young children to the exclusion of other developmental markers such as motor-behavioural and social-affective skills---all vital policy objectives along with reading and maths scores, if not more important, for kindergarten-entry age children (``interaction of the causal relationship with outcomes''). Lastly, the averaging procedure in calculating ATT washed out important differences across race and socio-economic groups, important factors reported by prior literatures as non-ignorable (``context-dependent mediation'').

\subsection{Construct Validity}

Construct validity concerns itself with the degree of agreement between the concept the researchers intended to understand (e.g., academic performance) and the procedure as well as instrument they employed to capture and measure such concept (e.g., sum scores in maths and reading tests). Amongst the various threats proposed by \textcite[][pp. 72--81]{shadish:2002}, \textcite{sullivan:2013} were particularly susceptible to ``inadequate explication of constructs'' and ``construct confounding''. The concept of academic performance can be thought as the end result of a sequence of social activities: academic input (I)--academic processing(P)--academic output(O). When academic scores were low, one is unable to ascertain whether it was the result of inferior teaching (xPO), lack of learning skills (IxO), or inability to demonstrate or document learning outcome to observers (IPx). Although both xPO and IPx may show up as low academic scores, the ``causal pathways'' cannot be more different---a situation not assisted by the ECLS-B early reading and math batteries used in \textcite{sullivan:2013} since none was designed to locate the source(s) of academic deficiency.

Construct confounding occurs when the concept under investigation has not been careful separated from other related concepts. \textcite{sullivan:2013} clearly wished to study ``academic performance'' of young children but such construct covary particularly strongly for this age group with attention span and sociability. It is not unreasonable to conjecture that recipients of the special education program may not develop the above-mentioned skills at the same pace as their counterparts. Effectively, the ECLS-B batteries employed by \textcite{sullivan:2013} were capturing young children's short attention spans and under-developed social skills and presenting them as inferior academic performance. Both inadequate explication of constructs and construct confounding have, therefore, weakened this study's construct validity, undermining its inference of ``special education causing lower academic performance''.

\subsection{Statistical Conclusion Validity}

By \textcite{cook:1979}, statistical conclusion validity refers to the appropriateness of statistical techniques employed by the researcher for the purposes of inferring whether the presumed independent and dependent variables indeed covary. The propensity score weighting technique employed by \textcite{sullivan:2013} successfully circumvented many pitfalls summarised by \textcite[][pp. 45--52]{shadish:2002} except for the ``restriction of range'' threat to statistical conclusion validity. Due to the necessity of constructing a region of common support, children are purposefully excluded from analyses if their probabilities of being accepted into the special education program fall outside of the 1\% to 82\% range. This practice is especially concerning for the above-82\% group since these are the young children with demonstrated need for urgent education assistance. Under the law of diminishing marginal returns, it is more than probable that it is this most-in-need group that would have responsed best and most rapidly to special education interventions. The wholesale omission of this positive outcome pool may have well contributed to the underestimation of the project effectiveness.

\section{Appropriateness of Methods}
%//mark Discussion of the extent to which the causal question has been appropriately answered by the study

\textcite{sullivan:2013} largely followed the propensity score anlysis procedure prescribed by \textcite{imbens:2015b} and \textcite{imbens:2015a} in assessing causal effects. At the first stage \textsc{design}, the authors established sufficient overlapping by discarding some units from the original sample in order to establish the region of common support; the second stage \textsc{supplementary analysis}, however, appeared to be lacking in \textcite{sullivan:2013} where the plausibility of unconfoundedness shall be further addressed through pseudo-average treatment effect on the pseudo-outcome for trimmed sample \parencite[see][pp. 383--384]{imbens:2015a}; such absence would cast doubt on any result in the third stage \textsc{analysis} over the source of average treatment effect.

One highlight on methodology is the Bayesian approach to AAT sampling weights. \textcite{sullivan:2013} correctly pointed out the ``curse of dimensionality'' when computing
$w_i$ for $D_i = 0$ cases and provided sufficient derivation through Bayes formula in reaching the form
\[ w_i = \frac{\p{ D=1 | z_i}}{1-\p{ D=1 | z_i}} \cdot \frac{\p{D=0}}{\p{D=1}},\ \text{for}\ D_i=0. \]
The authors, however, stayed short of advocating for a wider adoption of this approach to resampling weights but gave in to conventional literature in order to maximise comparability. This weighting formula overcome the peculiarity of the conventional scheme (only the first term in the formula above) summing to twice the size of the treated subsample and provided a more intuitive formation of summing to the sample size. Stronger advocacy can be expected from continuing research in popularising \poscite{sullivan:2013} weighting formulation.
\section{Conclusion}
%//mark Conclusion

