\begin{center}
    \textbf{Propensity Scores}
\end{center}

\begin{hangparas}{0.5in}{1}
    \fullcite{sullivan:2013}
\end{hangparas}

\section{Summary}
%//mark Summary of the article

\textcite{sullivan:2013} attempted to investigate the marginal benefit created by a special education program mandated by the US \textit{Individuals with Disabilities Education Act} (IDEA). The 1986 amendments to the IDEA legislation imposed legal requirement on the states to create a sequence of intervention programs targeting three stages of development for children with needs: 1) from birth to 2-year-old, 2) 3-year to kindergarten entry, and 3) kindergarten to 21-year-and-11-month-old. This paper focused on the middle segment of the intervention sequence named preschool special education services and followed children longitudinally through the delivery period. The study began by measuring children's probability of being admitted into the special education program using a logit regression, and at the end of the intervention period, compared the average performance scores in maths and reading of the children who actually received the preschool special education treatment against the average scores of those who have not. Using the propensity score weighting technique, the authors interrogated the counterfactual question of ``what if'' the treatment were never applied and reached an conclusion that children in treatment group would have been better off academically had they not received any intervention at all. This disappointing result lent itself to an existing literature of similar negative findings evaluating special education effectiveness, cautioning the rosy objectives of preschool special education services originally marketed by policy makers.

\section{Causal Question}
%//mark Identification of the causal question investigated in the study

Would the children who received special education services have been better off academically, on average, had they not received such services?

\section{Validity}
%//mark Discussion of validity aspects of the study: statistical conclusion validity, internal validity, construct validity, and external validity

\subsection{Construct Validity}

\subsection{Internal Validity}

\subsection{External Validity}

\subsection{Statistical Conclusion Validity}

\section{Appropriateness of Methods}
%//mark Discussion of the extent to which the causal question has been appropriately answered by the study

\section{Conclusion}
%//mark Conclusion
