\begin{center}
    \textbf{Instrumental Variables}
\end{center}

\begin{hangparas}{0.5in}{1}
    \fullcite{hanandita:2014}
\end{hangparas}

\bigskip

\section{Summary}
%//mark Summary of the article

\textcite{hanandita:2014} investigated the causal relationship between poverty and mental health decline using an instrumental variable (IV) approach in order to overcome the endogeneity problem. Using a sample size of 577,548 across 440 districts in Indonesia and precipitation anomaly as the IV, the authors were able to quantify the expenditure/income elasticity of mental disorders as $-0.62$---a result five times stronger than that of the non-IV approach and robust to various stress tests. Moreover, income inequality also appeared to carry explanatory power to mental health concerns in addition to that of poverty, suggesting both the position (quantity of income) and the shape (distribution of income) of the income curve as policy variables worth pursuing for the betterment of population mental welfare.

\section{Causal Question}
%//mark Identification of the causal question investigated in the study

Does poverty cause poor mental health?

\section{Validity}
%//mark Discussion of validity aspects of the study: statistical conclusion validity, internal validity, construct validity, and external validity

\subsection{Construct Validity}

\subsection{Internal Validity}

\subsection{External Validity}

\subsection{Statistical Conclusion Validity}

\section{Appropriateness of Methods}
%//mark Discussion of the extent to which the causal question has been appropriately answered by the study

The instrumental variable approach adopted by \textcite{hanandita:2014} served their research purpose (to overcome endogeneity problem) and claim (poverty causes mental health decline) well. The last paragraph in Section 1 of the paper paid particular justification to the key assumptions behind the IV method, namely relevance condition, validity condition and exclusion restriction and revisited the suitability of these assumptions in the third paragraph of Section 6, admitting that ``[t]he quality of an instrumental variabel estimation is only as good as its story'' (p. 65). Although untestable, the proposal put forward by the authors that precipitation anomaly was a random assignment procedure perfectly uncorrelated with the outcome variable (mental health condition) but covaried strongly with input variables (income) due to large proportion of the Indonesian labour force being employed in a rain-dependent agricultural sector, was a convincing one.

\section{Conclusion}
%//mark Conclusion

\textcite{hanandita:2014} had delivered a carefully designed study to the social science community. They elevated their research enquiry from a correlational endeavour to a causal one not only to satisfy one's methodological curiosity but to provide a conclusive response to the policy choice that ``if causal links between wealth and health were confirmed, society would likely benefit from more universal access to health care and redistributive economic policy. Yet, if such causal links were rebutted, resources would be better spent on influencing health knowledge, preferences, and ultimately the behavior of individuals.'' (Stowasser et al. (2011) as cited in \textcite{hanandita:2014}).