\begin{center}
    \textbf{Instrumental Variables}
\end{center}

\begin{hangparas}{0.5in}{1}
    \fullcite{hanandita:2014}
\end{hangparas}

\bigskip

\section{Summary}
%//mark Summary of the article

\textcite{hanandita:2014} investigated the causal relationship between poverty and mental health decline using an instrumental variable (IV) approach in order to overcome the endogeneity problem. Using a sample size of 577,548 across 440 districts in Indonesia and precipitation anomaly as the IV, the authors were able to quantify the expenditure/income elasticity of mental disorders as $-0.62$---a result five times stronger than that of the non-IV approach and robust to various stress tests. Moreover, income inequality also appeared to carry explanatory power to mental health concerns in addition to that of poverty, suggesting both the position (quantity of income) and the shape (distribution of income) of the income curve as policy variables worth pursuing for the betterment of population mental welfare.

\section{Causal Question}
%//mark Identification of the causal question investigated in the study

Does poverty cause poor mental health?

\section{Validity}
%//mark Discussion of validity aspects of the study: statistical conclusion validity, internal validity, construct validity, and external validity

Since the previous critique paper has documented a typology of validity, subsequent analyses would not repeat such content but focus on the application on the paper \textcite{hanandita:2014}.

\subsection{Statistical Conclusion and Internal Validity}

Statistical conclusion validity concerns itself with whether the presumed cause and effect covary and how strongly, whereas internal validity asks whether the observed covariation is causal in nature. \textcite{hanandita:2014} carefully avoided all threats to statistical conclusion validity put forward by \textcite[][Table 2.2, p. 45]{shadish:2002}. The endogeneity problem, for example, would have violated the Gauss-Markov assumption of $\E{\varepsilon_i | \m{x}_i}=0$; \textcite{hanandita:2014} not only restored independent error condition through the introduction of an IV (see, for example, \textcite{greene:2018} Chapter 8 for a technical discussion of IV), but also reported the magnitude of underestimation due to such assumption violation. By ensuring the IV to be uncorrelated with mental health but highly correlated with income, this study introduced an appropriate circuit breaker to the infinite feedback loop between poverty and mental health conditions, clearly suggesting the covariation between the two had indeed been causal and the arrow of causation points from income to mental health, not the other way around.

\subsection{Construct and External Validity}

Both construct and external validity deals with generalisation. In addressing the construct validity, Section 4.3 of \textcite{hanandita:2014} has been careful in distinguishing expenditure from income, and reported the observed deterioration in mental health condition as a response to reduction in consumption expenditure, in order to separate permanent income changes from intermittent income shock. Since this study used large dataset collected at national level, interactions of the causal relationship with both settings and outcomes can be minimum \parencite[see Table 3.2,][p. 87]{shadish:2002}, therefore delivering strong external validity.

\section{Appropriateness of Methods}
%//mark Discussion of the extent to which the causal question has been appropriately answered by the study

The instrumental variable approach adopted by \textcite{hanandita:2014} served their research purpose (to overcome endogeneity problem) and claim (poverty causes mental health decline) well. The last paragraph in Section 1 of the paper paid particular justification to the key assumptions behind the IV method, namely relevance condition, validity condition and exclusion restriction and revisited the suitability of these assumptions in the third paragraph of Section 6, admitting that ``[t]he quality of an instrumental variabel estimation is only as good as its story'' (p. 65). Although untestable, the proposal put forward by the authors that precipitation anomaly was a random assignment procedure perfectly uncorrelated with the outcome variable (mental health condition) but covaried strongly with input variables (income) due to large proportion of the Indonesian labour force being employed in a rain-dependent agricultural sector, was a convincing one.

The model building process was also appropriate. The authors ran their IV models against their baseline counterpart (i.e., models without IV); this comparison revealed a five-fold increase in the estimated effect of poverty on mental health due to the introduction of IV, incidentally revealing the magnitude of underestimation of the na{\"i}ve regression approach.

Other methodological considerations also enhanced \textcite{hanandita:2014}'s credibility. The authors explored both linear (linear and LPM) and non-linear (Poisson and Probit) configurations of their models to show the reported results were unlikely to be an mere artifect of the chosen functional forms. Correctional procedures such as the incorporation of sampling weights and clustering also safeguarded variance estimates. Centring of continuous variables such as log per capita household expenditure and the Gini coefficient also enhanced interpretability of the numeric results.

\section{Conclusion}
%//mark Conclusion

\textcite{hanandita:2014} had delivered a carefully designed study to the social science community. They elevated their research enquiry from a correlational endeavour to a causal one not only to satisfy one's methodological curiosity but to provide a conclusive response to the policy choice that ``if causal links between wealth and health were confirmed, society would likely benefit from more universal access to health care and redistributive economic policy. Yet, if such causal links were rebutted, resources would be better spent on influencing health knowledge, preferences, and ultimately the behavior of individuals.'' (Stowasser et al. (2011) as cited in \textcite{hanandita:2014}). The causal evidence presented by this paper would facilitate policy actions by updating scientific believes towards the former option and contribute to the betterment of mental health project in Indonesia and developing countries at large.