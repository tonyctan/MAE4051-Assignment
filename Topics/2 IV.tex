\begin{center}
    \textbf{Instrumental Variables}
\end{center}

\begin{hangparas}{0.5in}{1}
    \fullcite{hanandita:2014}
\end{hangparas}

\bigskip

\section{Summary}
%//mark Summary of the article

\textcite{hanandita:2014} investigated the causal relationship between poverty and mental health decline using an instrumental variable (IV) approach in order to overcome the endogeneity problem. Using a sample size of 577,548 across 440 districts in Indonesia and precipitation anomaly as the IV, the authors were able to quantify the income elasticity of mental disorders as $-0.62$---a result five times stronger than that of the non-IV approach and robust to various stress tests. Moreover, income inequality also appeared to carry explanatory power to mental health concerns in addition to that of poverty, suggesting both the position (quantity of income) and the shape (distribution of income) of the income curve as policy variables worth pursuing for the betterment of population mental welfare.

\section{Causal Question}
%//mark Identification of the causal question investigated in the study

\section{Validity}
%//mark Discussion of validity aspects of the study: statistical conclusion validity, internal validity, construct validity, and external validity

\subsection{Construct Validity}

\subsection{Internal Validity}

\subsection{External Validity}

\subsection{Statistical Conclusion Validity}

\section{Appropriateness of Methods}
%//mark Discussion of the extent to which the causal question has been appropriately answered by the study

\section{Conclusion}
%//mark Conclusion